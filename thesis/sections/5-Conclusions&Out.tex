{

\setlength{\parindent}{0pt}
\setlength{\parskip}{1em}

\chapter{Conclusion \& Outlook}

The objective of this thesis was to design, train and deploy a \gls{cnn} model for cloud segmentation on an embedded system \glsxtrlong{devboard}.
To complete the objective a configurable training and a deployment pipelines were implemented.
During this implementation phase, three major technical issues were discovered and successively solved,
they are documented as three milestones in \secshortref{chapter:implementation}.
Complete pipelines allowed to conveniently train, deploy and do the performance evaluation for five different \gls{cnn} networks,
designed in the scope of this work.
Particular attention was paid to tailoring these models for efficient inference on \gls{edgetpu}.
Difficuilties and caveats of conversion and compilation of trained models were overcame
and successful delegation of all tensor opertaions to \gls{edgetpu} is confirmed and documented.

Designed and evaluated \code{improvedCloudEdgeQ} architecture yields almost similar performance to the original \code{Cloud-Net} model\cite{CloudNet2019}.
\code{improvedCloudEdgeQ}, utilizing only \gls{rgb} \ensuremath{192\times192} patches as inputs, achieved 75.85\% Jaccard Index,
compared to \code{Cloud-Net}' 78.50\% on the same 20 scenes test set. This trades only a 3.38 \% performance decrease for a fast embedded system inference,
that does not requires \gls{nir} input information and processes one scene from Landsat 8 dataset in <> s.

There are several improvement suggestions, that can be made to different parts of the concluded work. These are stemming from the
ideas and insights gathered over the course of development process. Here are some of them.

An extension to 38 Cloud Dataset, already intoduced in \cite{CloudNetNew}, can be utilized for more extensive and effective training process.
Model's architecture improvement techniques can be gathered from \cite{CloudNetNew}, modified and ported on \gls{devboard},
utilizing provided pipelines from this work.


By loading the model to \gls{edgetpu} the off-chip memory used has to be reduced to the minimum for faster inference.
An error, outlined in \secshortref{sec:deployment} has to be investigated and eliminated to allow flawless postprocessing steps immediately after inference.
C++ inference implementation and delevopment experience allows further improvements to eventually adapt the deployment algorithms for uploading them to satellite,
taking space grade requirements into consideration.

Finally, it can be said that this work lays the foundation for future exploration of the feasibility of \gls{cots} components in satellites.
It allows the implementation of not only dummy tensor processing algorithms for radiation shielding tests,
but actually constitutes a versatile tool ready to be utilized and improved
for the development of sophisticated cloud segmentation algorithms for \gls{edgetpu} embedded systems.

\todo{mention binary crossentropy compared to softJaccardLoss! Maybe this in evaluation.}

}