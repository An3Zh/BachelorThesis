{

\setlength{\parindent}{0pt}
\setlength{\parskip}{1em}

\chapter{Conclusions \& Outlook}

The objective of this thesis was to design, train, and deploy a \gls{cnn} model for cloud segmentation on an embedded system \glsxtrlong{devboard}.
To achieve this objective, configurable training and deployment pipelines were implemented.
During implementation, three major technical issues were identified and successively addressed; they are documented as milestones in \secshortref{chapter:implementation}.
The complete pipelines enabled convenient training, deployment, and performance evaluation of five \gls{cnn} networks designed in this work.
Particular attention was paid to tailoring these models for efficient inference on the \gls{edgetpu}.
Difficulties and caveats in conversion and compilation were overcome, and the successful delegation of all tensor operations to the \gls{edgetpu} was confirmed and documented.

The \code{improvedCloudEdgeQ} architecture achieved performance comparable to the original \code{Cloud-Net} \cite{CloudNet2019}.
With \gls{rgb} inputs and \(\,192\times192\,\) patches, it reached a Jaccard Index of \(75.85\%\), versus \(\,78.50\%\) for \code{Cloud-Net} on the same 20 scene test set.
This corresponds to a modest \(2.65\) percentage points (\(3.38\%\) relative) decrease in performance in exchange for fast embedded inference that does not require \gls{nir} input
and processes one Landsat 8 scene in 216~s on the \gls{devboard}.

Several improvements are suggested, based on insights gathered during development.
An extension of the 38-Cloud dataset, as introduced in \cite{CloudNetNew}, could be used for more effective training.
Architecture improvements proposed in \cite{CloudNetNew} can be adapted and ported to the \gls{devboard} using the pipelines developed here.

When loading the model onto the \gls{edgetpu}, off-chip memory usage should be minimized to preserve inference speed.
The error outlined in \secshortref{sec:deployment} should be investigated and eliminated to allow post-processing immediately after inference.
The C++ inference implementation and the associated development experience enable further refinements,
ultimately supporting adaptation of the deployment algorithms for onboard satellite use under space-grade requirements.

Overall, this work lays a foundation for exploring the feasibility of \gls{cots} components in satellites.
Beyond simple tensor-processing payloads for radiation tests, the resulting pipeline and models constitute a
versatile toolchain for developing sophisticated cloud segmentation algorithms on \gls{edgetpu} based embedded systems.

}