{

\setlength{\parindent}{0pt}
\setlength{\parskip}{1em}

\chapter{Introduction}
\section{Motivation}

Electronic components in satellites and spacecraft are exposed to intense radiation, high-energy particles, and extreme temperature fluctuations.
To ensure reliability under these harsh conditions, manufacturers traditionally rely on certified space-grade components.
However, design, production, and certification of such components are costly and time\--con\-su\-ming. 

The AITHER project aims to investigate whether \gls{cots} components can be effectively used in nano- and microsatellites (10-100 kg).
Specifically, the project explores the feasibility of using \gls{cots}-based onboard architecture to perform reliable and computationally demanding tasks,
such as matrix operations in deep-learning based cloud segmentation, directly in space. 

As part of this initiative, the 10kg nanosatellite OOV-CUBE was launched into orbit on July 9th 2024.
Among its hosted payloads is the \gls{devboard} --- a single-board computer developed by Google with an embedded \gls{edgetpu} designed for fast,
low-power machine learning inference in constrained environments.

Alongside radiation shielding and thermal management strategies,
a technology demonstrator is being developed to assess the tolerance of these components to space radiation.
A central objective of the project is to determine not only how broadly \gls{cots} components can be applied in space systems,
but also whether complex image processing tasks can be carried out onboard.
This would reduce the need for data downlink and significantly improve the efficiency of Earth observation and communication missions.

\clearpage
\section{Objective}

While the satellite's structure was designed with radiation shielding and thermal management in mind,
the software for onboard inference still needs to be developed and uplinked into the orbit.
This thesis focuses on the implementation of \gls{cnn} for cloud segmentation from satellite imagery.
Although efficient \gls{cnn} architectures already exist and have been described in recent scientific literature,
porting such a model to an embedded platform introduces unique challenges. These include limitations in memory,
computation power of both \gls{cpu} and \gls{tpu} as well as model format compatibility.

The goal of this thesis is to design, train and deploy a \gls{cnn} model for cloud segmentation on the \gls{devboard}.
The core task is to adapt the model for efficient inference on the \gls{edgetpu},
overcoming hardware constraints while maintaining acceptable segmentation performance.
This work serves as a demonstration of the potential to carry out deep learning inference onboard a satellite using \gls{cots} hardware.

\section{Scope and Delimitations}

This work focuses on overcoming the implementation challenges of a deep learning algorithm on an embedded system.
The \code{Cloud-Net} \cite{CloudNet2019} architecture for binary cloud segmentation in satellite images will be modified
for deployment on the \glsxtrlong{edgetpu} located on the \glsxtrlong{devboard}.
The pipeline foundation will be laid to enable the transfer of other AI algorithms to the \gls{edgetpu} format in the future.
Predictions of a \gls{cnn}, adapted for embedded inference on the \gls{edgetpu}, will be evaluated.

Model training will be limited to the 38-Cloud dataset referenced in \cite{CloudNet2019}.
The \gls{rgb} and \gls{nir} channels of the input satellite imagery may be used.
Extensive hyperparameter tuning to achieve extraordinary performance in cloud segmentation will not be undertaken.
Software engineering requirements specific to satellite development are not considered.

}