{

\setlength{\parindent}{0pt}
\setlength{\parskip}{1em}

\section{Introduction}
\subsection{Motivation}

Electronic components in satellites and spacecraft are exposed to intense radiation, high-energy particles, and extreme temperature fluctuations. To ensure reliability under these harsh conditions, manufacturers traditionally rely on certified space-grade components. However, design, production, and certification of such components are costly and time\--con\-su\-ming. 

The AITHER project aims to investigate whether COTS (Commercial Off-The-Shelf) components can be effectively used in mano- and microsatellites (10-100 kg). Specifically, the project explores the feasibility of using COTS-based onboard architecture to perform reliable and computationally demanding tasks -- such as matrix operations in deep-learning based cloud segmenation -- directly in space. 

As part of this initiative, the 10kg nanosatellite OOV-CUBE was launched into orbit on July 9th 2024. Among its hosted payloads is the Coral Dev Board Mini – a single-board computer developed by Google with an embedded Edge TPU designed for fast, low-power machine learning inference in constrained environments.

Alongside radiation shielding and thermal managements strategies, a technology demonstrator is being developed to assess the tolerance of these components to space radiation (e.g. protons, gamma rays). A central objective of the project is to determine not only how broadly COTS components can be applied in space systems, but also whether complex image processing tasks can be carried out onboard. This would reduce the need for data downlink and significantly improve the efficiency of Earth observation and communication missions.

\subsection{Objective}

While the satellite's structure was designed with radiation shielding and thermal management in mind, the software for onboard inference still needs to be developed and uplinked in the orbit. This thesis focuses on the implementation of a convolutional neural network (CNN) for cloud segmentation from satellite imagery. Although lightweight CNN architectures already exist and have been described in recent scientific literature, porting such a model to an embedded platform introduces unique challenges. These include limitations in memory, computation power of both central processing unit (CPU) and tensor processing unit (TPU) as well as model format compatibility.

The goal of this thesis is to design, train and deploy a CNN model for cloud segmentation on the Coral Dev Board Mini. The core task is to adapt the model for efficient inference on the Edge TPU, overcoming hardware constraints while maintaining acceptable segmentation performance. This work serves as a demonstration of the potential to carry out deep learning inference onboard a satellite using COTS hardware.


}